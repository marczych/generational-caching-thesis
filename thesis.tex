\documentclass[12pt]{article}

\usepackage{ulem}
\usepackage{multicol}

\begin{document}

\title{\vfill Survey Paper Proposal: Web Application Caching}

\author{
By Marc Zych \vspace{10pt} \\
CSC 560: Grad Databases \vspace{10pt} \\
Dr. Alexander Dekhtyar \vspace{10pt} \\
}
\date{October 9, 2012}

\maketitle

% \begin{abstract}
% Stuff
% \end{abstract}

\thispagestyle{empty}
\newpage

%\tableofcontents

%\newpage
%\begin{multicols}{2}

\section{Topic Description}
A common method of scaling web applications is to use an in-memory cache that is shared between all web servers.
One such system is \texttt{memcached}.
This reduces server load and page response times by storing expensive calculations for later use.
To make caching as useful as possible, improving the application's cache hit rate is paramount.
What data is cached, how to determine the cache key, and how to handle updates are critical issues that directly affect the cache hit rate and how effective it is.

%\end{multicols}

\newpage
\nocite{*}
\bibliographystyle{IEEEannot}
\bibliography{thesis}
\end{document}
